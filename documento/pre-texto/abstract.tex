% Abstract
\begin{abstract}
% Diminuir espaçamento entre título e texto
\vspace{-1cm}
% Texto do resumo, em inglês: sem paragrafo, justificado, com espaçamento 1,5 cm
\onehalfspacing
\noindent
  This monograph conducted an experiment on the use of programming languages ​​Clojure and Scala in the Java virtual machine. The objective was to analyze metrics through the use of these languages ​​using the paradigm of functional programming in writing certain types of algorithms and that were compared to the algorithms written in the standard language of the Java virtual machine which is the Java programming language that it makes use of the paradigm of object-oriented programming. The benchmarking used as metrics to compare the number of lines of each algorithm and the execution time in milliseconds spent by each algorithm.

  After analyzing the data collected by the experiment proposed in this work, the results show that using functional programming languages ​​like Clojure or Scala in the Java virtual machine achieves better performance in algorithms that need doing calculations or iterate over lists of objects, and also to write less code to achieve the desired results.

% Espaçamento para as palavras-chave
\vspace*{.75cm}

% Palavras-chave: sem parágrafo, alinhado à esquerda
\noindent Keywords: Functional Programming, Object Oriented Programming, Java, Scala, Clojure, Java Virtual Machine. \\
% Segunda linha de palavras-chave, com espaçamento.
%\indent\hspace{1.4cm} Keyword.

\end{abstract}
