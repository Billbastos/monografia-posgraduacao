% Resumo
\begin{resumo}
% Diminuir espaçamento entre título e texto
\vspace{-1cm}

% Texto do resumo: sem paragrafo, justificado, com espaçamento 1,5 cm
\onehalfspacing

\noindent
  %objetivo e metodo
  Esta monografia realizou um experimento sobre o uso das linguagens de programação Clojure e Scala na máquina virtual Java. O objetivo foi analisar através de métricas o uso dessas linguagens que utilizam o paradigma da programação funcional na escrita de determinados tipos de algoritmos e que foram comparados com algoritmos escritos com a linguagem padrão da máquina virtual Java que é a linguagem de programação Java, e que faz uso do paradigma de programação orientado a objetos. A avaliação comparativa utilizou como métricas para comparação a quantidade de linhas de cada algoritmo, e o tempo em milissegundos de execução gasto por cada algoritmo.

  %resultado e conclusoes
  Após análise dos dados coletados pelo experimento proposto nesse trabalho, os resultados mostram que utilizando linguagens de programação funcional como Clojure ou Scala na máquina virtual Java, consegue-se obter um melhor desempenho em algoritmos que precisam fazer cálculos ou iterar sobre listas de objetos e também de escrever menos código para se obter os resultados desejados.

% Espaçamento para as palavras-chave
\vspace*{.75cm}

% Palavras-chave: sem parágrafo, alinhado à esquerda
\noindent Palavras-chave: Programação funcional, Programação Orientada a Objetos, Java, Scala, Clojure, máquina virtual Java.\\
% Segunda linha de palavras-chave, com espaçamento.
%\indent\hspace{2cm}Palavra.

\end{resumo}
