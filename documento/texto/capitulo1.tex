\chapter{Introdução}

\vspace{-1.9cm}

Desde o surgimento da máquina virtual Java, a linguagem de programação Java vem sendo utilizada para desenvolver sistemas para as mais diversas finalidades utilizando o paradigma de programação orientada a objetos, porém nos últimos anos surgiram outras linguagens de programação que conseguem gerar bytecode Java e que por sua vez podem ser executadas dentro da máquina virtual Java. Algumas dessas linguagens tem como base o paradigma da programação funcional, sendo que as que mais se destacam são as linguagens Clojure e Scala.

Um ponto interessante a se considerar entre as linguagens de programação Clojure e Scala e a própria linguagem de programação Java é que mesmo utilizando paradigmas de programação distintos ainda existe interoperabilidade entre elas e que com isso é possível compartilhar algoritmos e invocar códigos escritos entre estas linguagens de programação.

\section{Justificativa}

  Embora a primeira vista pareça interessante utilizar em um mesmo projeto linguagens funcionais como Scala ou Clojure em conjunto com a linguagem Java, faz-se necessário um estudo para avaliar e descobrir em quais circunstâncias algoritmos escritos através da programação funcional possuem vantagem sobre algoritmos que produzem os mesmos resultados, mas que foram escritos através da programação orientada a objetos.

  A principal justificativa para a realização desta pesquisa é demonstrar o uso da programação funcional dentro da máquina virtual Java através das linguagens Clojure e Scala, e as vantagens que pode-se obter ao optar pelo uso dessas linguagens.

\section{Objetivos}

  \subsection{Objetivo Geral}

    O principal objetivo desse trabalho é introduzir o paradigma da programação funcional, sua essência e a sua utilização dentro da plataforma Java, assim como demonstrar a viabilidade de uso do paradigma funcional na escrita de algoritmos e também expor sua simplicidade, melhor legibilidade e desempenho para determinadas tarefas.

  \subsection{Objetivos Específicos}

    \begin{compactitem}
      \item Contextualizar os princípios do paradigma da programação funcional e da programação orientada a objetos.
      \item Introduzir o uso da programação funcional dentro da plataforma Java por meio das linguagens Clojure e Scala;
      \item Introduzir o uso da interoperabilidade entre linguagens de programação dentro da plataforma Java;
      \item Como aplicar testes comparativos de desempenho e de como avaliar a legibilidade em algoritmos escritos em Clojure, Java e Scala;
    \end{compactitem}

\section{Problema}

  Observa-se a necessidade do mercado pelo rápido desenvolvimento de aplicações e em conjunto uma grande dificuldade dos desenvolvedores para acompanhar essa velocidade na implementação de algoritmos que proporcionem ao mesmo tempo simplicidade na codificação e ganho de desempenho. Algumas linguagens de programação não oferecem uma flexibilidade na codificação de algoritmos, gerando muito mais código e consequentemente dificultando o seu entendimento e a sua facilidade de manutenção.

\section{Metodologia}

  A modalidade de trabalho proposta para esta pesquisa é utilizar o formato de experimento, para assim conseguir realizar testes com a utilização de dois paradigmas de programação distintos e de mudanças em como algoritmos são escritos mesclando o uso de diferentes paradigmas de programação em uma única plataforma.

  O método de pesquisa aplicado será uma avaliação comparativa entre dois paradigmas de programação.

  O método de abordagem aplicado nesta pesquisa será uma abordagem indutiva e qualitativa, para que a pesquisa explore e apresente uma análise com base nos dados obtidos pela avaliação comparativa.

\section{Organização do Documento}

  O primeiro capítulo contém uma breve introdução sobre esta monografia, a descrição das motivações, interesses e objetivos utilizados para compor esta monografia.

  No segundo capítulo apresenta-se o que são paradigmas de programação e descreve-se também o paradigma de programação funcional e o paradigma de programação orientada a objetos.

  O terceiro capítulo apresenta a máquina virtual Java, o bytecode Java e a interoperabilidade entre linguagens de programação na plataforma Java.

  No quarto capítulo apresenta-se as linguagens de programação Clojure, Scala e Java.

  O quinto capítulo apresenta os programas utilizados nos experimentos.

  O sexto capítulo apresenta a execução dos programas apresentados no capítulo anterior, assim como os resultados obtidos por eles.

  O sétimo capítulo apresenta a conclusão obtida através do experimento realizado nesse trabalho.
