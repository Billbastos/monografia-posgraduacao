\chapter{Experimentos Propostos}

\vspace{-1.9cm}

Neste capítulo, são apresentados os programas utilizados no experimento, sendo estes escritos nas linguagens de programação Java, Clojure e Scala seguindo esta mesma sequência. Todos os programas serão seguidos de uma breve explicação e comentário. Todos os programas geram o tempo de execução em milissegundos.

\section{Somar Números Inteiros}

  Este primeiro programa tem por finalidade somar os primeiros 1000000 números inteiros.

  \subsection{Java}

     Implementação em Java:

    \begin{lstlisting}[language=Java, mathescape=false]
    public class SomarNumeros {

      public static void main(String[] args) {
        double tempoInicial = System.nanoTime();

        long somatoria = 0;
        for (long i = 1; i < 1000000; i++) {
          somatoria += i;
        }

        double tempoFinal = System.nanoTime();

        System.out.println("Resultado: " + somatoria);
        System.out.println("Tempo (ms): " + ((tempoFinal - tempoInicial) / 1000000));
      }

    }
    \end{lstlisting}

  \subsection{Clojure}

    Implementação em Clojure:

    \begin{lstlisting}[language=Clojure, mathescape=false]
    (ns somarnumeros.core
      (:gen-class))

    (defn -main
      [args]

      (def tempoInicial (double (. System (nanoTime))))

      (def resultado (loop [somar (long 0) inicio (long 1)]
        (if (< inicio 1000000) (recur (+ somar inicio) (inc inicio)) somar)))

      (def tempoFinal (double (. System (nanoTime))))

      (println (format "Resultado: %d" resultado))
      (println (format "Tempo (ms): %.6f" (/ (- tempoFinal tempoInicial) 1000000))))
    \end{lstlisting}

  \subsection{Scala}

    Implementação em Scala:

    \begin{lstlisting}[language=Scala, mathescape=false]
    object SomarNumeros extends App {

      val tempoInicial: Double = System.nanoTime

      def somar(inicio: Long, total: Long, resultado: Long = 0): Long =
        if (inicio >= total) resultado else somar(inicio + 1, total, inicio + resultado)

      val resultado = somar(1, 1000000)
      val tempoFinal: Double = System.nanoTime

      println(s"Resultado: $resultado")
      println(s"Tempo (ms): ${((tempoFinal - tempoInicial) / 1000000)}")

    }
    \end{lstlisting}


\section{Manipulação de Listas}

  Este programa tem por finalidade gerar um lista com números inteiros de 1 a 1000000, em seguida o programa deve filtrar todos os números pares dessa lista.

  \subsection{Java}

    Implementação em Java:

    \begin{lstlisting}[language=Java, mathescape=false]
    import java.util.ArrayList;
    import java.util.List;

    public class FiltrarNumerosPares {

      public static void main(String[] args) {
        double tempoInicial = System.nanoTime();

        List<Long> resultado = new ArrayList<Long>();

        for (long numero = 1; numero <= 1000000; numero++) {
          if (numero % 2 == 0) {
            resultado.add(numero);
          }
        }

        double tempoFinal = System.nanoTime();

        System.out.println("Resultado: " + resultado.size());
        System.out.println("Tempo (ms): " + ((tempoFinal - tempoInicial) / 1000000));
      }

    }
    \end{lstlisting}

  \subsection{Clojure}

    Implementação em Clojure:

    \begin{lstlisting}[language=Clojure, mathescape=false]
    (ns manipulacaodelistas.core
      (:gen-class))

    (defn -main
      [args]

      (def tempoInicial (double (. System (nanoTime))))

      (def numeros (range 1 1000001))
      (def resultado (filter #(= 0 (rem % 2)) numeros))

      (def tempoFinal (double (. System (nanoTime))))

      (println (format "Resultado: %d" (.size resultado)))
      (println (format "Tempo (ms): %.6f" (/ (- tempoFinal tempoInicial) 1000000))))
    \end{lstlisting}

  \subsection{Scala}

    Implementação em Scala:

    \begin{lstlisting}[language=Scala, mathescape=false]
    object FiltrarNumerosPares extends App {

      val tempoInicial: Double = System.nanoTime

      val numeros = 1 to 1000000
      val resultado = numeros.view.filter(_ % 2 == 0)

      val tempoFinal: Double = System.nanoTime

      println(s"Resultado: ${resultado.size}")
      println(s"Tempo (ms): ${((tempoFinal - tempoInicial) / 1000000)}")

    }
    \end{lstlisting}

\section{Cálculo Fatorial}

  Este programa tem por finalidade calcular o fatorial do número 50.

  \subsection{Java}

    Implementação em Java:

    \begin{lstlisting}[language=Java, mathescape=false]
    public class CalcularFactorial {

      public BigInteger factorial(BigInteger f, BigInteger n) {
        if (n.compareTo(new BigInteger("1")) == 0)
          return f;
        else
          return factorial(f.multiply(n), n.subtract(new BigInteger("1")));
      }

      public static void main(String[] args) {
        double tempoInicial = System.nanoTime();

        CalcularFactorial c = new CalcularFactorial();
        BigInteger resultado = c.factorial(new BigInteger("1"),  new BigInteger("50"));

        double tempoFinal = System.nanoTime();

        System.out.println("Resultado: " + resultado);
        System.out.println("Tempo (ms): " + ((tempoFinal - tempoInicial) / 1000000));
      }

    }
    \end{lstlisting}

  \subsection{Clojure}

    Implementação em Clojure:

    \begin{lstlisting}[language=Clojure, mathescape=false]
    (ns calculofatorial.core
      (:gen-class))

    (defn -main
      [args]

      (def tempoInicial (double (. System (nanoTime))))

      (defn factorial [f n]
        (if (= n 1)
          f
          (factorial (* f n) (dec n))))

      (def resultado (factorial 1N 50N))

      (def tempoFinal (double (. System (nanoTime))))

      (println (format "Resultado: %d" (biginteger resultado)))
      (println (format "Tempo (ms): %.6f" (/ (- tempoFinal tempoInicial) 1000000))))
    \end{lstlisting}

  \subsection{Scala}

    Implementação em Scala:

    \begin{lstlisting}[language=Scala, mathescape=false]
    object CalcularFactorial extends App {

      val tempoInicial: Double = System.nanoTime

      def factorial(f: BigInt, n: BigInt): BigInt = {
        if (n == 1) f else factorial((f * n), (n - 1))
      }

      val resultado = factorial(1, 50)
      val tempoFinal: Double = System.nanoTime

      println(s"Resultado: $resultado")
      println(s"Tempo (ms): ${((tempoFinal - tempoInicial) / 1000000)}")

    }
    \end{lstlisting}
