\chapter{Conclusão}

\vspace{-1.9cm}

  A linguagem de programação Java é uma linguagem muito popular e muito utilizada para se construir programas das mais diversas finalidades. Com o passar do tempo, mesmo com a evolução tanto da linguagem Java como da máquina virtual Java, é possível notar que em determinados tipos de algoritmos a linguagem Java atrapalha ou impede o desenvolvedor de escrever programas de forma mais concisa e mantendo um bom desempenho, principalmente em códigos executados de forma concorrente ou que precisam iterar sobre coleções de objetos realizando agrupamentos ou filtros.

  Para desenvolvedores com bom conhecimento sobre a linguagem de programação Java e que não possuem conhecimento sobre nenhum dialeto Lisp, aprender e entender um programa escrito em Scala torna-se mais fácil e intuitivo, pois embora Scala faça um forte uso do paradigma funcional, ela ainda utiliza alguns conceitos do paradigma de orientação a objetos. Um outro ponto importante é que Scala também é estaticamente tipada assim como a linguagem Java, além de que quando compara com a linguagem Clojure, Scala ainda possui uma sintaxe mais próxima da linguagem Java.

  Ao contrário da linguagem Java, as linguagens Clojure e Scala são pouco conhecidas e ainda pouco utilizadas, embora seu uso vem sendo encorajado devido a grandes empresas como Twitter, LinkedIn, Facebook e Netflix anunciarem que vem utilizando essas linguagens com grande êxito em seus sistemas.

  Em face de tudo o que foi exposto anteriormente, conclui-se que é possível obter mais desempenho com programas mais concisos fazendo uso de linguagens que utilizam o paradigma de programação funcional dentro da máquina virtual Java, assim como foi visto nos capítulos introdutórios sobre o paradigma funcional e sobre as linguagens Clojure e Scala.
